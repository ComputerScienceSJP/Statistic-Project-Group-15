
\documentclass[12pt,a4paper]{article}
\usepackage[margin=1in]{geometry}
\usepackage{graphicx}
\usepackage{booktabs}
\usepackage{float}
\usepackage{longtable}
\usepackage{hyperref}
\usepackage{setspace}
\usepackage{csquotes}
\usepackage{siunitx}
\usepackage{enumitem}
\usepackage{caption}
\usepackage{xcolor}

\hypersetup{colorlinks=true, linkcolor=blue, urlcolor=blue, citecolor=blue}
\doublespacing

\begin{document}

% ----------------------------
% COVER / TITLE PAGE
% ----------------------------
\begin{titlepage}
\centering
\begin{figure}[H]
  \centering
  \includegraphics[width=0.3\textwidth]{images.png}
\end{figure}
{\Large \textbf{University of Sri Jayewardenepura}\\[0.3em] Faculty of Computing\par}
\vspace{1.5cm}
{\large \textbf{CCS\,|\,CIS2042 Statistical Distribution \& Inferences}\par}
\vspace{2.0cm}
{\LARGE \bfseries Statistical Analysis of Shift to Solar Power\\[0.25em] in Colombo and Gampaha (2022--2024)\par}
\vspace{1.2cm}
\begin{tabular}{ll}
\textbf{Course:} & CCS|CIS2042 Statistical Distribution \& Inferences \\
\textbf{Project:} & Statistical analyze of Shift to Solar Power in Colombo and Gampaha (2022--2024) \\
\textbf{Group 15 :} & W K S D Walallawita -- FC223044 -- Interpretation 1 \\
                 & U K D S Sandaruwan -- FC221006 -- Interpretation 2 \\
                 & S V D De Silva -- FC223033 -- Interpretation 3 \\
                 & A G Sanduwara -- FC223013 -- Interpretation 4 \\
                 & E M U S Edirisinghe -- FC223016 -- Interpretation 5 \\
\textbf{Lecturer:} & Ms. D N M Hettiarachchi \\
\textbf{Date:} & \today
\end{tabular}
\vfill
\end{titlepage}

% ----------------------------
% ABSTRACT
% ----------------------------
\begin{abstract}
This research investigates the adoption of solar power in the Colombo and Gampaha districts of Sri Lanka during the period 2022--2024. The study analyzes the transition from grid electricity to solar power at the household and business levels and identifies the socio\hyp{}economic and demographic factors influencing this shift. Using survey data collected via Google Forms, we apply descriptive and inferential statistical techniques including chi\hyp{}square tests, t\hyp{}tests, confidence intervals, and logistic regression. The findings aim to provide evidence on adoption patterns, economic implications (bill reductions and investment levels), and perceived barriers, informing both academic discussions and policy practice.
\end{abstract}

\clearpage
\tableofcontents
\clearpage

% ----------------------------
% INTRODUCTION
% ----------------------------
\section{Introduction}

As the world shifts away from fossil fuels, solar power is emerging as a key renewable energy source due to its sustainability and falling installation costs. In Sri Lanka, rising electricity demand—driven by urban growth and population increases—makes the transition to clean energy a national priority.

The government has introduced initiatives like \textit{Soorya Bala Sangramaya} to promote rooftop solar, and many households are now considering solar as a viable alternative to grid electricity.

Colombo and Gampaha, being highly urbanized and energy-intensive districts, provide an ideal setting to explore solar adoption trends. Between 2022 and 2024, adoption has grown, possibly influenced by income, incentives, and public awareness. However, challenges remain, and a data-driven understanding of adoption patterns is needed.

This study aims to:
\begin{itemize}[noitemsep,topsep=0pt]
    \item Analyze solar adoption trends in Colombo and Gampaha (2022--2024),
    \item Assess how socio-economic and demographic factors influence adoption,
    \item Test for differences in adoption between districts,
    \item Offer insights to guide national solar energy strategies.
\end{itemize}

Using survey data and statistical methods like hypothesis testing and estimation, this research provides evidence to support renewable energy policy and adoption at both local and national levels.


% ----------------------------
% LITERATURE REVIEW
% ----------------------------
\section{Literature Review}
\begin{itemize}[noitemsep,topsep=0pt]
    \item \textbf{Policy and incentives in Sri Lanka:} Analyses of national programs (e.g., Soorya Bala Sangramaya) and how incentives shape household decisions.
    \item \textbf{Socio\hyp{}economic determinants:} Links between income, electricity prices, awareness, and renewable adoption in South Asia.
    \item \textbf{Statistical approaches:} Use of chi\hyp{}square tests, t\hyp{}tests/ANOVA, confidence intervals, and logistic regression in technology adoption studies.
    \item \textbf{Cost\hyp{}benefit and satisfaction:} Evidence of post\hyp{}installation bill reductions and user satisfaction with rooftop solar.
\end{itemize}

% ----------------------------
% METHODOLOGY
% ----------------------------
\section{Methodology}
\subsection{Data Source and Period}
Primary data were collected via a structured Google Form from households and businesses in Colombo and Gampaha. Data collection targets at least 150 valid responses (70 collected to date), capturing adoption between 2022 and 2024.

\subsection{Survey Variables}
\textbf{Demographics:} Age group (18--25, 26--35, 36--50, 51+), gender, district (Colombo, Gampaha), occupation, household income bracket, type of premises (household, business, both).\\
\textbf{Electricity Usage:} Monthly bill (LKR categories), monthly units consumed, peak usage periods (time of day), high\hyp{}consumption devices (AC, refrigerator, heater, washing machine, microwave/oven, desktops, TVs, etc.).\\
\textbf{Solar Users:} Installation year (2022/2023/2024), reasons for switching (cost, environment, power cuts, investment, trend), initial investment (LKR categories), satisfaction (1--5), post\hyp{}adoption bill change, recommendation (yes/no/maybe).\\
\textbf{Non\hyp{}Users:} Barriers (high cost, lack of knowledge, performance uncertainty, space, not a priority), future plans, and factors encouraging adoption (subsidies, price drops, awareness, warranties, recommendations).

\subsection{Sampling and Ethics}
A combination of convenience and snowball sampling was used. Participation was voluntary; responses are anonymized and used solely for academic purposes in accordance with institutional guidelines.

\subsection{Statistical Methods (R)}
Analyses were conducted in R using descriptive statistics, confidence intervals, and hypothesis testing.

% ----------------------------
% RESULTS & INTERPRETATION 1
% ----------------------------
\section{Results and Interpretations}

\subsection{Interpretation 1: Proportion of Solar Adopters}

\subsubsection*{Results}

Out of the 200 valid responses collected, \textbf{26 participants (13\%)} reported that they had installed a solar power system at their premises, while 174 had not. A one-sample proportion test was conducted to estimate the adoption rate of solar systems among households in the Colombo and Gampaha districts.

The estimated sample proportion was $\hat{p} = 0.13$. The 95\% confidence interval for the true population proportion was calculated as:
\[
\text{CI}_{95\%} = [0.0882,\ 0.1865]
\]

This means that we are 95\% confident that the true adoption rate in the population lies between approximately \textbf{8.82\% and 18.65\%}.

\begin{figure}[H]
  \centering
  \includegraphics[width=0.65\textwidth]{4f049239-464b-4a00-b180-7f3a291ad883.png}
  \caption{Distribution of solar power adoption among 200 surveyed households.}
  \label{fig:solar-adoption}
\end{figure}

Only 13\% of the surveyed respondents have adopted solar power, confirming that the majority still rely on grid electricity. The narrow confidence interval supports the reliability of this estimate. These results point to a need for improved awareness, better affordability, or stronger support schemes to increase adoption. Overall, more focused programs are needed to help households transition to clean energy in Sri Lanka.



\subsection{Interpretation 2: Post-Adoption Bill Reduction (Proxy Scale)}

\subsubsection*{Results}

For participants who had adopted solar systems, we measured the perceived change in their monthly electricity bill using a qualitative proxy scale. The responses were transformed into numeric scores using the following coding:

\begin{itemize}[noitemsep, topsep=0pt]
    \item Decreased significantly = 2
    \item Decreased slightly = 1
    \item No change = 0
    \item Increased = -1
\end{itemize}

Using these values, a one-sample $t$-test was performed to check if the average score was greater than zero, which would suggest that bills generally decreased after adopting solar.

The mean score was approximately 0.86, with a 95\% confidence interval of [0.72,\ 1.00]. The $t$-test result was statistically significant ($p$-value $<$ 0.001), showing strong evidence of bill reduction among adopters.

\begin{figure}[H]
  \centering
  \includegraphics[width=0.65\textwidth]{8bcbaae5-fa9c-4b63-9cfc-10b1879511b5.png}
  \caption{Reported monthly electricity bill change among solar adopters.}
  \label{fig:bill-reduction}
\end{figure}

These results confirm that most people who installed solar systems feel that their electricity bills have gone down. The high average score and the tight confidence interval suggest that this benefit is common. Most responses were in the “decreased slightly” or “decreased significantly” categories, supporting the idea that solar power is helping users save money on energy in everyday life.


\subsection{Interpretation 3: Correlation Between Usage and Monthly Bill}

\subsubsection*{Results} \\

To examine the relationship between monthly electricity usage (in units) and the 
average monthly electricity bill (in LKR), Pearson’s correlation and a simple linear 
regression were performed. Midpoint values were calculated for each response 
range to obtain numeric data.

The correlation test indicated a positive relationship between usage and bill, 
with a Pearson correlation coefficient of $r \approx 0.42$ (p-value $< 0.05$). 
This suggests that households with higher electricity usage tend to report 
higher monthly bills.

A simple linear regression model was also fitted:

\[
\text{Bill} = a + b \times \text{Usage}
\]

where $a$ represents the intercept and $b$ the slope. The estimated slope was 
positive, confirming that as usage increases, the electricity bill rises. The 
coefficient of determination ($R^2$) was approximately 0.18, meaning about 18\% 
of the variation in bills can be explained by usage differences.


The scatterplot with a fitted regression line (Figure~\ref{fig:usage_vs_bill}) 
shows this positive trend. While the correlation is moderate, it reflects a clear 
economic link: higher electricity consumption leads to higher monthly expenses. 
This finding is important because it highlights how solar adoption could bring 
greater benefits to high-usage households, making solar a financially attractive 
alternative for managing rising energy costs.

\begin{figure}[h]
    \centering
    \includegraphics[width=0.7\linewidth]{9e5cf5db-f01b-41a1-88ad-f5690448fb04.png}
    \caption{Scatterplot of monthly electricity usage (units) versus bill (LKR) with fitted regression line.}
    \label{fig:usage_vs_bill}
\end{figure}



\subsection{Interpretation 4: Estimation of Average Monthly Electricity Bill}

\subsubsection*{Results}

This analysis estimates the average monthly electricity bill based on responses given in predefined ranges (e.g., ``Rs.\ 3000--4000''), which were converted into midpoint values for calculation. The sample includes both solar and non-solar users.

The estimated average monthly bill was approximately \textbf{Rs.\ 4800}, with a 95\% confidence interval of \textbf{Rs.\ 4500 to Rs.\ 5100}, based on a one-sample $t$-test. This narrow interval shows that most households fall within a consistent billing range.

To visualize this, the original response ranges were plotted as shown below.

\begin{figure}[H]
  \centering
  \includegraphics[width=0.8\textwidth]{d66c0896-c977-4ccd-8416-935a96a83674.png}
  \caption{Monthly Electricity Bill Ranges Reported by Respondents}
  \label{fig:monthly-bill-bar}
\end{figure}

Most respondents reported monthly bills between Rs.\ 3000 and Rs.\ 8000, with Rs.\ 4800 as the average. These figures give a clear idea of electricity spending in Colombo and Gampaha. Understanding this spending range helps evaluate how affordable solar power really is. Since the range is narrow, the estimate is reliable for guiding policies and support programs.

\subsection{Interpretation 5: Electricity Usage Distribution}

This analysis investigates the distribution of monthly electricity usage (in units) among respondents. Since responses were given in usage ranges (e.g., ``61 -- 90''), midpoints were calculated to approximate numerical values.

A histogram was plotted using these values. Two curves were added for comparison:
\begin{itemize}
    \item \textbf{Blue Line:} Actual density from the collected data
    \item \textbf{Red Dashed Line:} Fitted normal distribution using sample mean and standard deviation
\end{itemize}

\begin{figure}[H]
\centering
\includegraphics[width=0.7\textwidth]{91fa456e-4b9f-4bd0-80c2-6a1addf13484.png}
\caption{Distribution of Monthly Electricity Usage (in Units)}
\end{figure}

The histogram shows a roughly bell-shaped pattern, with peaks around 75 to 105 units. While the normal curve does not perfectly fit, it provides a good approximation overall. The result suggests that most households use a moderate and consistent amount of electricity, likely due to similar appliance usage or family sizes. These findings give useful insights into energy demand levels across the sample.



\section{Discussion}

This section interprets the statistical findings from the five analyses, drawing connections between the observed trends and possible real-world explanations. These insights aim to support practical understanding of solar adoption behavior in Colombo and Gampaha districts.

\subsection*{Interpretation 1: Proportion of Solar Adopters}

The analysis of adoption proportions showed that only 13\% of the respondents had installed solar systems. Despite nationwide efforts to promote renewable energy, such a low adoption rate implies several persistent barriers—possibly financial, informational, or infrastructural. The confidence interval ranging from 8.8\% to 18.7\% confirms this adoption is still in early phases. This suggests a need for further engagement campaigns or financial incentives to boost household transitions to solar energy.

\subsection*{Interpretation 2: Post-Adoption Bill Reduction}

The proxy scale-based t-test for bill changes among adopters revealed a statistically significant reduction in monthly electricity expenses. Most adopters reported slight to significant decreases in their bills, with an average proxy score of 0.86. This validates a key benefit of solar energy—lower utility costs—and supports its promotion as an economically attractive solution. The clear financial return observed here could be used in awareness efforts and incentive policy justifications.

\subsection*{Interpretation 3: Correlation Between Usage and Monthly Bill}

The correlation and regression analysis revealed a positive relationship between monthly electricity usage (units) and the monthly bill (LKR). The moderate Pearson correlation shows that as usage increases, bills also rise. The fitted regression line confirmed this trend, with a positive slope indicating that each additional unit of electricity contributes to higher costs. Although the $R^2$ value suggests that usage alone does not explain all the variation in bills, the link is clear and statistically meaningful. This result highlights the practical reality that households with higher usage face greater financial pressure, and therefore stand to benefit the most from adopting solar power as a cost-saving strategy.


\subsection*{Interpretation 4: Estimation of Monthly Electricity Bill}

The average monthly electricity bill was estimated at approximately Rs.~4800, with a narrow confidence interval. This estimation provides a baseline for understanding energy costs and decision-making thresholds for solar adoption. As electricity bills are a motivating factor for switching, this estimate can help model payback periods and assess subsidy effectiveness. Moreover, the narrow confidence band suggests homogeneity in respondent consumption levels, which may guide future segmentation of outreach efforts.

\subsection*{Interpretation 5: Distribution of Monthly Electricity Usage}

The histogram and overlaid normal curve suggested a moderately normal distribution of monthly electricity usage, centered around 80--100 units. Some deviations from perfect normality (such as peaks) were noted, possibly due to similar appliance usage patterns among respondents. This analysis not only supports planning for standard solar system sizing but also reveals that most households have mid-level consumption, further validating solar as a suitable solution for urban energy needs.

\section{Conclusion}

This study statistically examined the shift to solar power in Colombo and Gampaha from 2022 to 2024 using data from 200 participants. Key findings revealed that solar adoption remains modest at 13\%, suggesting significant potential for further growth. Among adopters, strong evidence showed reductions in electricity bills, confirming the economic benefit of solar systems in urban settings.

No significant difference in adoption rates between districts suggests that unified regional policies may be effective. The estimated average monthly bill of Rs.~4800 and the roughly normal distribution of electricity usage offer useful benchmarks for planning and policy.

These insights highlight solar power as a scalable and cost-effective energy solution. However, barriers to adoption—such as awareness, upfront cost, or trust in technology—still require attention.

\textbf{Limitations:} The study used self-reported data from only two districts, which may limit broader applicability. Future research could include a wider geographical scope and more detailed operational metrics.

In summary, the findings support solar energy expansion in Sri Lanka and offer a model for similar studies aiming to evaluate renewable energy adoption and its impacts.
\section*{References}

\begin{itemize}
    \item Ministry of Power and Energy. (2022). \textit{Soorya Bala Sangramaya – National Solar Rooftop Initiative}. Government of Sri Lanka. Retrieved from \url{https://powermin.gov.lk/}
    
    \item Perera, D., \& Gunawardena, A. (2020). Statistical modeling of solar energy adoption in Sri Lanka. \textit{Journal of Sustainable Energy}, 15(2), 45–59.
    
    \item World Bank. (2021). \textit{Expanding Solar Power in South Asia: A Regional Study}. Retrieved from \url{https://www.worldbank.org/}
    
    \item Jayasundara, J. M., \& Wijesinghe, A. (2021). Factors affecting household solar adoption in Sri Lanka. \textit{Renewable Energy Journal}, 29(3), 211–225.
\end{itemize}

\clearpage
\appendix
\section*{Appendix}

\subsection*{A. Sample R Code Used in Analysis}

\begin{verbatim}
# Load dataset
df <- read.csv("Shift to Solar Power in Colombo and Gampaha (2022-2024).csv")

# Interpretation 1: Proportion of Solar Adopters
adopted <- df$`Have.you.installed.a.solar.power.system.at.your.premises.` == "Yes"
prop.test(sum(adopted), length(adopted), conf.level = 0.95)

# Interpretation 2: Bill Reduction (Proxy Scale)
map <- c("Decreased significantly"=2, "Decreased slightly"=1, "No change"=0, "Increased"=-1)
df$Proxy <- map[df$`How.did.your.monthly.electricity.bill.change.after.switching.to.solar.`]
t.test(df$Proxy, mu=0, alternative="greater")

# Interpretation 3: Chi-Square for District vs Adoption
table <- table(df$District, adopted)
chisq.test(table)

# Interpretation 4: Monthly Bill Estimation
midpoints <- c("3000 - 4000" = 3500, "4000 - 5000" = 4500, "5000 - 6000" = 5500)
df$MidBill <- midpoints[df$`Average.Monthly.Electricity.Bill..in.LKR.`]
t.test(df$MidBill)

# Interpretation 5: Electricity Usage Distribution
range_to_mid <- function(x) {
  parts <- unlist(strsplit(x, " - "))
  mean(as.numeric(parts))
}
df$UsageMid <- sapply(df$`Average.Monthly.Electricity.Usage..in.Units.`, range_to_mid)
hist(df$UsageMid, prob=TRUE, col="lightgray", main="Usage Histogram")
lines(density(df$UsageMid), col="blue", lwd=2)
curve(dnorm(x, mean(df$UsageMid), sd(df$UsageMid)), col="red", lty=2, add=TRUE)
\end{verbatim}

\subsection*{B. Example Survey Questions}

\begin{itemize}
  \item What is your district? (Colombo / Gampaha)
  \item Have you installed a solar power system at your premises?
  \item What was your average monthly electricity bill before solar panels ? (Range)
  \item How satisfied are you with your solar panel system (1–5 scale)?
  \item How did your electricity bill change after switching to solar?
  \item Would you recommend solar power to others?
\end{itemize}

\subsection*{C. Transformed Midpoint Categories}

\begin{table}[H]
\centering
\caption{Converted Midpoint Values for Monthly Electricity Bill}
\begin{tabular}{ll}
\toprule
\textbf{Original Range} & \textbf{Converted Midpoint (LKR)} \\
\midrule
Rs.\ 3000--4000 & 3500 \\
Rs.\ 4000--5000 & 4500 \\
Rs.\ 5000--6000 & 5500 \\
Rs.\ 6000--8000 & 7000 \\
\bottomrule
\end{tabular}
\end{table}

\end{document}
